\documentclass[a4paper,11pt,final]{article}

% Packages
\usepackage[french]{babel}
\usepackage[latin1]{inputenc}
\usepackage[T1]{fontenc}
\usepackage{lmodern}

\usepackage{vmargin}
\usepackage{graphicx}
\usepackage{color}
\usepackage{url}
\usepackage{listings}
\usepackage{tikz,pgf}
\usepackage{array}
\usepackage{multicol}
\usepackage{microtype}
\usepackage{background}

% Marges
\setpapersize[landscape]{A4}
\setmargins{3cm}{0cm}{25.5cm}{17cm}{0cm}{2cm}{0cm}{0cm}

% Style
\pagestyle{empty}
\urlstyle{sf}
\setlength{\parindent}{0mm}

% Couleurs
\definecolor{ecamorange}{rgb}{0.92,0.45,0.02} % #ec7405
\definecolor{ecamred}{rgb}{0.68,0.06,0.28} % #ae0f48

%%% Style des listings
\lstdefinelanguage{bash}
{morekeywords=[1]{git},
morekeywords=[2]{origin},
morekeywords=[3]{config,clone,push},
morestring=[b]{"},}

\lstset{basicstyle=\scriptsize\tt}
\lstset{keywordstyle=[1]\color[rgb]{0,0,1}\bfseries}
\lstset{keywordstyle=[2]\color[rgb]{0.6,0,0}\bfseries}
\lstset{keywordstyle=[3]\color[rgb]{0,0.6,0}\bfseries}
\lstset{identifierstyle=\color{black}}
\lstset{commentstyle=\color[rgb]{0,0.6,0.6}}
\lstset{stringstyle=\color[rgb]{0.44,0.47,1}}
\lstset{showstringspaces=false}
\lstset{showtabs=false}
\lstset{tabsize=3}
\lstset{extendedchars=true}
\lstset{breaklines=true}
\lstset{postbreak={}}
\lstset{breakautoindent=true}
\lstset{breakindent=0pt}
\lstset{xleftmargin=0.1cm}
\lstset{xrightmargin=1mm}
\lstset{frame=tblr,rulecolor=\color[gray]{.4}}
\lstset{captionpos=b}
\lstset{aboveskip=0.5cm,belowskip=0cm}
\lstset{numbers=left}
\lstset{numberstyle=\tiny}
\lstset{columns=fixed}
\lstset{escapechar=�}
\lstset{language=python}

% Nouvelles commandes
\newcommand{\code}[1]{{\small\texttt{#1}}}

% D�but du document

%Template du document ce repetant sur chaque page
\backgroundsetup
{
scale=1,
angle=0,
opacity=1,
hshift=-265,
vshift=320,
contents={\begin{tikzpicture}[overlay]
		\draw[draw=none,fill=ecamorange] (-3.5,2) rectangle (-0.9,-20);
		\node at (-2,1) {\includegraphics[width=2cm]{ECAM-logo.pdf}};
		\node[rotate=90,scale=2.5,color=white] at (-2,-15.5) {\sf\bfseries Cheat Sheet};
		\node[scale=2,color=ecamred,anchor=west] at (-0.3,0.8) {\sf\bfseries Commandes Bash de base};
		\node at (25.5,-18.75) {\includegraphics[width=2cm]{by-nc-nd.pdf}};
\end{tikzpicture}}
}

\begin{document}
	\begin{multicols}{2}
	
		\bigskip
		\begin{tabular}{>{\tt\footnotesize}p{4cm}p{7.5cm}}
			\multicolumn{2}{l}{{\Large\color{ecamred}\sf G�n�ral}\medskip}\\						
			clear & efface le contenu de la fen�tre \\[1mm]
			history & affiche l'historique des derni�res commandes \\[1mm]
			exit & quitte le terminal
		\end{tabular}
		
		% - - - - - - - - - - - - - - - - - - - - - - - - - - - - - - - - - - - - - - - - - - - - - - - - - - - - - - -
		
		\bigskip
		\begin{tabular}{>{\tt\footnotesize}p{4cm}p{7.5cm}}
			\multicolumn{2}{l}{{\Large\color{ecamred}\sf Information syst�me}\medskip}\\			
			whoami & affiche l'identit� de l'utilisateur courant \\[1mm]
			uptime & information sur la dur�e de fonctionnement \\[1mm]
			date & affiche la date courante \\[1mm]
			cal & affiche le calendrier du mois courant \\[1mm]
			man <cmd> & affiche l'aide de la commande \code{cmd} \\
			& (\textit{quitter l'aide avec la touche \code{q}}) \\[1mm]
			uname -a & information sur le syst�me d'exploitation \\[1mm]
			w & liste les utilisateurs connect�s \\[1mm]
			df & affiche l'utilisation des disques \\
			|\quad df -h & tailles lisibles par un humain
		\end{tabular}

		
		% - - - - - - - - - - - - - - - - - - - - - - - - - - - - - - - - - - - - - - - - - - - - - - - - - - - - - - -
		
		\bigskip
		\begin{tabular}{>{\tt\footnotesize}p{4cm}p{7.5cm}}
			\multicolumn{2}{l}{{\Large\color{ecamred}\sf Navigation}\bigskip}\\
			pwd & affiche le dossier courant \\[2mm]
			ls & liste le contenu du dossier courant \\
			|\quad ls <path> & liste le contenu du dossier \code{path} \\
			|\quad ls -a & affiche tous les fichiers et dossiers \\
			|\quad ls -lh & liste d�taill�e et tailles lisibles par un humain \\[2mm]
			cd & renvoi � votre r�pertoire de connexion\\			
			|\quad cd <path> & change le dossier courant en \code{path} \\
			|\quad cd .. & remonte dans le dossier parent
		\end{tabular}
		
		% - - - - - - - - - - - - - - - - - - - - - - - - - - - - - - - - - - - - - - - - - - - - - - - - - - - - - - -
	
		\bigskip
		\begin{tabular}{>{\tt\footnotesize}p{4cm}p{7.5cm}}
		 	\multicolumn{2}{l}{{\Large\color{ecamred}\sf Affichage du contenu d'un fichier}\medskip}\\
			cat <path> & affiche le contenu du fichier \code{path} \\[2mm]
			more <path> & permet l'affichage page par page du fichier \code{path} incluant une navigation basique \\[2mm]
			less <path> & idem \code{more }\code{path} mais incluant une navigation plus compl�te \\[2mm]
			head <path> & affiche les 10 premi�res lignes du fichier \code{path} \\
			|\quad head -n <nb> <path> & affiche les \code{nb} premi�res lignes du fichier \code{path} \\[2mm]
			tail <path> & affiche les 10 derni�res lignes du fichier \code{path} \\
			|\quad tail -n <nb> <path> & affiche les \code{nb} derni�res lignes du fichier \code{path}
		\end{tabular}	
		
		% - - - - - - - - - - - - - - - - - - - - - - - - - - - - - - - - - - - - - - - - - - - - - - - - - - - - - - -		
		

		\bigskip
		\begin{tabular}{>{\tt\footnotesize}p{4cm}p{7.5cm}}
			\multicolumn{2}{l}{{\Large\color{ecamred}\sf Manipulation de fichiers}\medskip}\\
			mkdir <path> & cr�e un nouveau dossier \code{path} \\[2mm]
			touch <path> & cr�e un nouveau fichier vide \code{path} \\[2mm]
			cp <path1> <path2> & copie le fichier \code{path1} vers le fichier \code{path2} \\
			|\quad cp <path1> <path2>/ & copie le fichier \code{path1} dans le dossier \code{path2} \\
			|\quad cp -R <path1> <path2> & copie le dossier \code{path1} vers le dossier \code{path2} \\[2mm]
			mv & (lorsque \code{path2} existe d�j�)\\
			|\quad mv <path1> <path2>/ & d�place le fichier \code{path1} dans le dossier \code{path2} \\
			|\quad mv <path1> <path2> & �crase le fichier \code{path2} par \code{path1} \\
			|\quad mv <path1> <path2> & d�place le dossier \code{path1} dans \code{path2} \\[2mm]
			mv & (lorsque \code{path2} n'existe pas)\\
			|\quad mv <path1> <path2> & renomme le fichier ou dossier \code{path1} en \code{path2} \\	[2mm]
			rmdir <path> & supprime le dossier \code{path} si celui-ci est vide\\[2mm]
			rm <path> & supprime le fichier \code{path} \\
			|\quad rm -i <path> & demande confirmation avant suppression \\
			|\quad rm -R <path> & supprime le dossier \code{path} \\[2mm]
			find <path> -name <name> & cherche le fichier \code{name} dans le dossier \code{path}
		\end{tabular}
		
		% - - - - - - - - - - - - - - - - - - - - - - - - - - - - - - - - - - - - - - - - - - - - - - - - - - - - - - -
		
		\bigskip
		\begin{tabular}{>{\tt\footnotesize}p{4cm}p{7.5cm}}
		\multicolumn{2}{l}{{\Large\color{ecamred}\sf Gestion des permissions}\medskip}\\	
		chmod & permet de changer les permissions d'un fichier\\[1mm]
		|\quad chmod +w <path> & ajoute les droits en �criture au fichier \code{path} � tout les utilisateur\\
		|\quad chmod g-x <path> & retire les droits d'ex�cution au fichier \code{path} pour le groupe seulement\\
	 	 |\quad chmod UGA <path> & d�finit touts les droits sous forme octal au fichier \code{path} (ex : chmod 755 <path>)
	 	 \begin{itemize}
	 	 \item U : droits de l'utilisateur
	 	 \item G : droits du groupe
	 	 \item	 A : droits des autres
		\end{itemize}	 	  \\
		chown & permet de changer le propri�taire d'un fichier\\[2mm]
		chgrp & permet de changer le groupe auxquelles appartient le fichier\\
		\end{tabular}
		
		% - - - - - - - - - - - - - - - - - - - - - - - - - - - - - - - - - - - - - - - - - - - - - - - - - - - - - - -		
		
		\bigskip
		\begin{tabular}{>{\tt\footnotesize}p{4cm}p{7.5cm}}
		\multicolumn{2}{l}{{\Large\color{ecamred}\sf Redirection}\medskip}\\	
		<cmd> > <path> & redirige le sortie standard de la commande \code{cmd} vers le fichier \code{path}\\
		<cmd> ?> <path> & redirige la sortie standard \textbf{et} la sortie d'erreurs de la commande \code{cmd} vers le fichier 				\code{path}\\
		<cmd> 1> <path1> 2> <path2>& redirige la sortie standard vers le fichier \code{path1} et la sortie d'erreurs vers le fichier 		\code{path2}\\
		\end{tabular}
		
		% - - - - - - - - - - - - - - - - - - - - - - - - - - - - - - - - - - - - - - - - - - - - - - - - - - - - - - -
		
		\bigskip
		\begin{tabular}{>{\tt\footnotesize}p{4cm}p{7.5cm}}
		\multicolumn{2}{l}{{\Large\color{ecamred}\sf Gestion des processus}\medskip}\\			
			ps & liste statique des processus en cours d'ex�cution de l'utilisateur courant \\
			|\quad ps -a & liste les processus de tous les utilisateurs \\
			|\quad ps -e & liste tous les processus \\
			|\quad ps -f & affiche toutes les informations des processus \\[2mm]
			top & liste dynamique des processus en cours d'ex�cution (\textit{quitter la liste avec la touche \code{q}}) \\[2mm]
			kill <pid> & tue le processus num�ro \code{pid} \\
			|\quad kill -9 <pid> & tue violemment le processus \\
		\end{tabular}
		
		% - - - - - - - - - - - - - - - - - - - - - - - - - - - - - - - - - - - - - - - - - - - - - - - - - - - - - - -
		
		\bigskip
		\begin{tabular}{>{\tt\footnotesize}p{4cm}p{7.5cm}}
			\multicolumn{2}{l}{{\Large\color{ecamred}\sf Divers}\medskip}\\
			diff <path1> <path2> & affiche la diff�rence entre les deux fichiers \\[1mm]
			wc <path> & affiche le nombre de lignes, de mots et de caract�res du fichier \code{path} \\[1mm]
			lpr <path> & imprime le fichier \code{path} \\[1mm]
			lpq & affiche les files d'impression \\[1mm]
			wget <url> & t�l�charge le fichier \code{url} \\[1mm]
			ping <url> & envoie une requ�te ping au serveur \code{url} \\[1mm]
			ifconfig & montre la configuration des interfaces r�seau\\[2mm]
			cat > <path> & permet d'envoyer un texte dans le fichier \code{path} (ctrl+d pour enregistrer) \\[1mm]
			cat >> <path> & permet d'envoyer un texte dans le fichier \code{path} � la suite du texte d�j� pr�sent.\\
		\end{tabular}
	\end{multicols}
\end{document}